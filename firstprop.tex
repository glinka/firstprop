\documentclass[11pt]{article}
\setlength{\textheight}{240mm}
\addtolength{\topmargin}{-15mm}
\setlength{\textwidth}{155mm}
\setlength{\oddsidemargin}{5mm}
\usepackage{graphicx, subfig}
\graphicspath{ {./figures/} }
\pagestyle{plain}
\begin{document}
\title{Dynamics of networks: generation, dimensionality reduction, and coarse-grained evolution of graphs}
\author{\LARGE Proposed by Alexander Holiday\vspace{3mm}\\\large under the supervision of\vspace{3mm}\\\LARGE Professor Yannis Kevrekidis}
\date{11/01/2013}
\maketitle

\begin{figure}[h]
  \centering
  \includegraphics[width=80mm]{princeton_shield}
\end{figure}

\section{Introduction}

From collaborations among movie stars \cite{barbasi} to gene interactions in \texit{C. elegans}\cite{harvardCelegans}, network science has found a vast array of applications in the past two decades. This is largely a result of the generality of the network framework: myriad systems are readily adapted to description as interacting bodies; the body (e.g. a city, person, or protein) forms a node in the network, and the interactions between them (e.g. via highways, Facebook friendships, or biological suppresion) create the edges. Thus, one may usefully apply the same abstraction to study such disparate topics as coupled oscillators and the spread of opinions in a society. \\
However, 

%\bibliographystyle{abbrv}
%\bibliography{firstprop_bib.bib}
\end{document}

